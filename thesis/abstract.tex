\phantomsection
\addcontentsline{toc}{chapter}{\bf ABSTRACT}

\begin{center}
    \large{\bf ABSTRACT}
\end{center}

Continuous glucose monitoring (CGM) systems have been highlighted as an important component of optimal glycemic control in diabetic patients (Lee, Probst, Klonoff, \&\ Sode, 2021). The commercial CGM technique currently includes the implantation of a sensor (Keenan, Mastrototaro, Voskanyan, \&\ Steil, 2009). Its minimally intrusive nature prevents glucose monitoring from becoming ubiquitous. Raman spectroscopy, for example, has been investigated as a non-invasive method of measuring hyperglycemia in vivo. Wearable (constant, non-invasive, ubiquitous) self-monitoring blood glucose (SMBG) is therefore achievable. However, the development of wearable Raman-based SMBGs has received little attention. This development is fraught with difficulties. First, the best measurement locations to directly evaluate glucose scattering (wrist, forearm, nail fold, fingertip, and thenar) are unknown. Although previous research demonstrated excellent accuracy of glucose prediction from all five sites, this high accuracy may be due to the utilization of the whole spectrum, which may contain undesired signal aberrations that correlate with hyperglycemia (Kang et al., 2020).
Second, we ran multiple trials from the previous study at measuring sites, but the resultant spectrum data was confusing. We went back to the foundation experiment, the calibration data from liquid samples, to overcome this difficulty and ensure that the experiment we devised was compatible with our Raman instrument. 
Third, appropriate features (engineering) are underutilized. Although previous work offered feature engineering strategies such as principal component analysis (Li et al., 2019) or protein/hemoglobin normalization (Kang et al., 2020; Shao et al., 2012), a lack of formal comparison makes understanding what works challenging. Along with that, a wearable SMBG prototype is being built and tested.
